\documentclass[a4paper, 10pt]{ustl-tdtp}
%\usepackage[utf8]{inputenc}  
%\usepackage[T1]{fontenc}
\usepackage{fontspec}
\usepackage{listings}  
\usepackage{version}
\usepackage{tikz}


%\usepackage[a4paper]{geometry}


\etablissement{\ustl}
\formation{DUT info 2ème année}
\matiere{Structures de données}
\titre{TD : Tables de hachage}
\date{\annee{2018}--\annee{2019}}
%\enseignant{}

%\includeversion{solution}
\excludeversion{solution}

\lstset{
    breaklines=true
}

\newcommand{\case}{}%\rule[-.40cm]{0pt}{1cm}}
\newcommand{\ident}[3]{(\texttt{#1},~\texttt{#2},~\texttt{#3})}
\newcommand{\rien}{}

\parindent 0cm
\begin{document}
\maketitle
\thispagestyle{empty}


\subsection*{Exercice 1}

~\\ \textbf{Question 1:}
Quelle est la définition d'une fonction de hachage ?

\begin{solution}
{\color{red}
La caractéristique principale d'une fonction de hachage est
  de produire un condensé de l'objet passé en paramètre. Ce condensé
  est de taille fixe, dont la valeur diffère suivant la fonction
  utilisée. Cette fonction doit se calculer efficacement. 
}
\end{solution}

~\\ \textbf{Question 2:}
Expliquez rapidement ce qu'est un sondage linéaire dans une table
de hachage. Quel est le problème du sondage linéaire ?

\begin{solution}
{\color{red}
Cette méthode permet de gérer les collisions dans une table
  de hachage : si on a une collision, on parcourt le tableau à partir
  de l'indice h(k) jusqu'à trouver une case libre. Les indices
  parcourus peuvent être écrits sous la forme suivante, avec M la
  taille du tableau, h la fonction de hachage et i le nombre de
  sondages effectués:
$$ indice = \left( h(k) + i \right) ~mod~ M .$$

Le problème de cette méthode est la formation de clusters de
  valeurs, ce qui augmente le coût des traitements.  
}
\end{solution}

 ~\\ \textbf{Question 3:} Nous considérons maintenant une table de
hachage de taille 11, initialement vide. Les collisions sont gérées
par sondage linéaire. Ajoutez successivement les valeurs suivantes à
la table de hachage. \textbf{Pour chaque élément, vous préciserez à
  quel index, il doit s'insérer, ainsi que les étapes de ce calcul.}
\begin{center}
\begin{tabular}{|c|c|l|}
\hline
Élément & Valeur de hachage & index de sondage\\
\hline 
A & 59 &\\
B & 7 &\\
C & 28 &\\
D & 21 &\\
E & 17 &\\
F & 95 &\\
G & 50 &\\
\hline
\end{tabular}
\end{center}

\begin{solution}
{\color{red}
Le premier index de sondage est le reste de la division euclidienne de
la valeur de hachage par la taille du tableau, à savoir 11. Si il y a
collision, on sonde linéairement à droite par pas de 1.


\begin{center}
\begin{tabular}{|c|c|l|}
\hline
Élément & Valeur de hachage & index de sondage\\
\hline 
A & 59 & 4\\
B & 7 & 7\\
C & 28 & 6 \\
D & 21 & 10\\
E & 17 & 6 $\rightarrow$ 7 $\rightarrow$ 8 \\
F & 95 & 7 $\rightarrow$ 8 $\rightarrow$ 9 \\
G & 50 & 6 $\rightarrow$ 7 $\rightarrow$ 8 $\rightarrow$ 9 $\rightarrow$
         10 $\rightarrow$ 0 \\
\hline
\end{tabular}
\end{center}

\textit{ Pour un résultat final :
\begin{center}
\begin{tabular}{|c|c|c|c|c|c|c|c|c|c|c|}
0&1&2&3&4&5&6&7&8&9&10\\
\hline
G&&&&A&&C&B&E&F&D\\
\hline
\end{tabular}
\end{center}
}
}
\end{solution}

~\\ \textbf{Question 4:} Puis supprimez la valeur B en expliquant
votre démarche.

\begin{solution}
{\color{red}
On supprime la valeur en la remplaçant par une valeur du
  cluster ayant une valeur de hachage inférieur ou égale. On répète
  cette opération jusqu'à la fin du cluster.

\textit{ Pour un résultat final :
\begin{center}
\begin{tabular}{|c|c|c|c|c|c|c|c|c|c|c|}
0&1&2&3&4&5&6&7&8&9&10\\
\hline
&&&&A&&C&E&F&G&D\\
\hline
\end{tabular}
\end{center}
}
}
\end{solution}


\subsection*{Exercice 2}

~\\ \textbf{Question 1:} Nous considérons une table de hachage en adressage ouvert de taille 11, et la fonction de hachage $h(k) = k~mod~11$. Insérez successivement les clés 11, 22, 31, 4, 15, 28, 17, 88 et 59 dans la table en utilisant un sondage linéaire. \textbf{Expliquez pour chaque clé le calcul de l'index d'insertion.}

\begin{solution}
{\color{red}
\textit{ Résultat final :
\begin{center}
\begin{tabular}{|c|c|c|c|c|c|c|c|c|c|c|}
0&1&2&3&4&5&6&7&8&9&10\\
\hline
11&22&88&&4&15&28&17&59&31&\\
\hline
\end{tabular}
\end{center}
}
}
\end{solution}

~\\ \textbf{Question 2:} Même question en résolvant les collisions par sondage quadratique.

\begin{solution}
{\color{red}
%On utilise la fonction de sondage quadratique $i + i^2$.
\textit{ Résultat final :
\begin{center}
\begin{tabular}{|c|c|c|c|c|c|c|c|c|c|c|}
0&1&2&3&4&5&6&7&8&9&10\\
\hline
11&22&&88&4&15&28&17&&31&59\\
\hline
\end{tabular}
\end{center}
}
}
\end{solution}

~\\ \textbf{Question 3:} Même question en résolvant les collisions par chaînage.

\begin{solution}
{\color{red}
\textit{ Résultat final :
\begin{center}
\begin{tabular}{|c|c|c|c|c|c|c|c|c|c|c|}
0&1&2&3&4&5&6&7&8&9&10\\
\hline
11|22|88&&&&4|15|59&&28|17&&&31&\\
\hline
\end{tabular}
\end{center}
}
}
\end{solution}

\subsection*{Exercice 3}

~\\ \textbf{Question 1:} Quels seraient les attributs d'une classe
implémentant un ensemble (\texttt{Set<E>}) grâce à une table de hachage
en adressage ouvert ?
\begin{solution}
{\color{red}
\begin{lstlisting}[language=Java]
public class MyHashSet<E> implements Set<E>{
    private E[] tab; // container of elements
    private int n; //number of elements
}
\end{lstlisting}
}
\end{solution}

~\\ \textbf{Question 2:} Quel serait le problème si nous voulons faire
un ensemble contenant des valeurs de type \texttt{int}  ou
\texttt{double}  ?

\begin{solution}
{\color{red}
Il faudrait considérer une valeur pour coder un emplacement vide dans
le tableau (par exemple : \texttt{Integer.MAX_VALUE} et \texttt{Double.MAX_VALUE}).
}
\end{solution}


~\\ \textbf{Question 3:} Programmez les algorithmes d'insertion et de suppression dans la table de hachage en adressage ouvert (dans le cas du sondage linéaire).

\begin{solution}
{\color{red}
\begin{lstlisting}[language=Java]
private int next(int i){
    return i==table.length-1?0:i+1;
}

public boolean add(E e){
    int indexPrimaire = e.hashCode() \% table.length;
    int i = indexPrimaire;
    while(table[i]!=null 
           && !table[i].equals(e)
           && next(i)!=indexPrimaire){
        i=next(i);
    }
    if(table[i]!=null){
        if(table[i].equals(e))
            return false;
        throw new IllegalStateException();
    }
    table[i] = e;
    return true;
}
\end{lstlisting}
}
\end{solution}

\end{document}

