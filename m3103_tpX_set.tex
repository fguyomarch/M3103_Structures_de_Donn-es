\documentclass[iutinfo,a4paper,nocorrections,10pt]{ustl-tdtp}
%\usepackage[utf8]{inputenc}
%\usepackage[a4paper]{geometry}

\etablissement{\ustl}
\formation{DUT info 2ème année}
\matiere{Structures de données}
\titre{TP 4 : Implémentation d'un Ensemble}
\date{\annee{2017}--\annee{2018}}

\parindent 0cm
\begin{document}
\includeversion{solution}
%\excludeversion{solution}

\maketitle
\thispagestyle{empty}

Le but de ce TP est de créer une classe \texttt{SortedArraySet<E>} qui implémente \texttt{SortedSet}. 

\section{Constructeurs}

Pour cette classe,nous choisissons comme conteneur un tableau, avec comme attribut additionnel un entier désignant le nombre d'élément de notre collection.

Comme on souhaite que le tableau soit trié, nous allons également prévoir un attribut de type \texttt{Comparator<? super E>}, qui peut permettre de comparer de objets de type E grâce à la méthode \texttt{compare(E e1, E e2)}. Cette référence doit être définie comme \texttt{final}.

~\\ \question Ajoutez les constructeurs suivants: \texttt{SortedArraySet<E>()}, et  
\texttt{SortedArraySet<E>(Comparator<? super E> comparator)} 
%\texttt{SortedArraySet<E>(Collection<? extends E) collection)} 

~\\ \question Programmez une méthode \texttt{compare(E e1, E e2)} qui compare les deux éléments passés en paramètre. Cette méthode qui respecte la convention usuelle du Java pour les comparaisons, utilise le \texttt{Comparator} s'il n'est pas \texttt{null} (s'il a été instancié par le constructeur). Si le \texttt{Comparator} est null, la comparaison se base sur l'ordre naturel des éléments comparables E, c'est-à-dire sur la méthode \texttt{compareTo}. Si aucune de ces deux possiblités n'est disponible, la méthode soulève une exception.

~\\ \question Programmez les méthodes \texttt{isEmpty},  \texttt{size}, \texttt{toString} et \texttt{clone}.

~\\ \question Programmez une méthode privée\footnote{Pourquoi faut-il absolument qu'elle soit privée ?} \texttt{indexOf} qui retourne l'index du premier élément supérieur ou égal à l'élément passé en paramètre (et pas -1 s'il n'est pas présent !). Cette méthode tirera bien évidemment parti du fait que le tableau est trié\footnote{Vive le retour de la dichotomie...} ! 

~\\ \question Programmez une méthode privée \texttt{insert(int index, E element)}, qui insère un élément E à la position index. Programmez à présent la méthode publique \texttt{add(E element)} qui insère un élément à sa position dans le tableau.  

~\\ \question Créer un main pour tester votre travail: créer un SortedArraySet<String> sans Comparateur spécifié, ajouter-y les éléments "Bonjour", "test", "abc", "Abajour", "lapin", et afficher le SortedArraySet ainsi obtenu. 

~\\ \question On souhaite tester le comportement de la classe en utilisant un Comparator à la place de l'ordre naturel. Créer une classe MonComparateur<String> qui implémente l'interface Comparator<String> avec  un main pour tester votre travail: créer un SortedArraySet<String> sans Comparateur spécifié, ajouter-y les éléments "Bonjour", "test", "abc", "Abajour", "lapin". 




et \texttt{remove(int index)}. Ces deux méthodes qui modifient le conteneur à partir d'un index, seront très pratiques pour programmer l'interface.







~\\ \question Programmez le reste des méthodes de l'interface \texttt{SortedSet<E>}. Pour vos tests, vous pouvez utiliser la classe de tests fournie avec le sujet du TP.

\begin{correction}
% Non vérifiée ;-)
{\color{red}
\begin{lstlisting}[language=Java]
    private int compare(E e1, E e2){
        if (comparator != null)
            return comparator.compare(e1, e2);
        return ((Comparable<E> e1).compareTo(e2);
    }
\end{lstlisting}
}

\end{correction}

\end{document}

