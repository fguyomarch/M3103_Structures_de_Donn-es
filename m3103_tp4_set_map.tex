\documentclass[iutinfo,a4paper,10pt]{ustl-tdtp}
%\usepackage[utf8]{inputenc}
%\usepackage[a4paper]{geometry}

\etablissement{\ustl}
\formation{DUT info 2ème année}
\matiere{Structures de données}
\titre{TP 4 : Ensembles et Tables de Correspondance}
\date{\annee{2018}--\annee{2019}}

\parindent 0cm
\begin{document}
%\includeversion{solution}
%\excludeversion{solution}

\maketitle
\thispagestyle{empty}

\section{Interface \texttt{Set}}
\sloppy

\question Dans la méthode \texttt{main} d'une classe \texttt{ExoSet}, créez deux ensembles {\it ens1} et {\it ens2} d'\texttt{Integer} en utilisant l'implémentation \texttt{HashSet} de l'interface \texttt{Set}. Initialiser ces ensembles avec quelques entiers choisis aléatoirement.

~\\ \question Écrivez une méthode \texttt{public static void afficher(Set
  ens)} permettant l'affichage d'un ensemble \texttt{ens} en
commençant par le nombre d'éléments contenus dans \texttt{ens} entre
crochets, puis chacun des éléments de cet ensemble entre accolades. Par exemple, l'ensemble $A=\{8,1,4,2\}$ aura comme affichage: 
\begin{verbatim}
  [4]{8, 1, 4, 2}
\end{verbatim}

Pour les méthodes suivantes, vous veillerez à ne pas modifier les ensembles passés en paramètres.

\question Créez une méthode \texttt{public static Set<Integer> inter(Set<Integer> set1, Set<Integer> set2)} qui renvoie un ensemble contenant l'intersection de deux ensembles en paramètres sans utiliser la méthode \texttt{retainAll} de \texttt{Set}.

~\\ \question Créez une méthode \texttt{public static Set<Integer> union(Set<Integer> set1, Set<Integer> set2)} qui renvoie un ensemble contenant l'union de deux ensembles sans utiliser la méthode \texttt{addAll} de \texttt{Set}.

~\\ \question Créez une méthode \texttt{public static boolean
  isIn(Set<Integer> set1, Set<Integer> set2)} qui renvoie vrai si
l'ensemble \texttt{set1} est inclus dans l'ensemble \texttt{set2} sans utiliser la méthode \texttt{containsAll} de \texttt{Set}.

~\\ \question Implémentez les méthodes de test \texttt{testInter}, \texttt{testUnion} et \texttt{testIsIn} dans une classe de test \texttt{ExoSetTest} qui vérifient le bon comportement des 3 méthodes décrites plus haut. 

\section{Implémentation de l'interface \texttt{SortedSet}}

\question Créez une classe \texttt{SortedArraySet<E>} qui implémente \texttt{SortedSet}. Nous choisissons pour celle-ci comme conteneur, un tableau trié, avec comme attributs additionnels, un entier désignant le nombre d'élément de notre collection, ainsi qu'une référence \texttt{comparator}\footnote{\texttt{comparator} sera \texttt{null} si les éléments sont triés suivant leur ordre naturel} sur un \texttt{Comparator}\footnote{Le type exact est \texttt{Comparator<? super E>}.}. Cette référence doit être définie comme \texttt{final}. Le tableau aura une taille pas défaut de 16.

~\\ \question Ajoutez les constructeurs suivants: \texttt{SortedArraySet<E>()}, 
\texttt{SortedArraySet<E>(Comparator<? super E> comparator)}, 
\texttt{SortedArraySet<E>(Collection<? extends E) collection)} 

~\\ \question Programmez une méthode \texttt{compare(E e1, E e2)} qui compare les deux éléments passés en paramètre. Cette méthode qui respecte la convention usuelle du Java pour les comparaisons, utilise le \texttt{comparator} s'il n'est pas \texttt{null}, sinon elle utilise l'ordre naturel des éléments. Si aucune de ces deux possibilités n'est disponible, la méthode soulève une exception de type \texttt{ClassCastException}.

~\\ \question Programmez les méthodes \texttt{isEmpty},  \texttt{size} et \texttt{clone}. Après avoir programmé \texttt{clone}, vous pouvez déclarer votre classe comme \texttt{Clonable}.

~\\ \question Programmez une méthode privée\footnote{Pourquoi faut-il absolument qu'elle soit privée ?} \texttt{indexOf} qui retourne l'index du premier élément supérieur ou égal à l'élément passé en paramètre (et pas -1 s'il n'est pas présent !). Cette méthode tirera bien évidemment parti du fait que le tableau est trié\footnote{Vive le retour de la dichotomie...} ! 

~\\ \question Programmez deux méthodes privées \texttt{insert(int index, E element)} et \texttt{remove(int index)}. Ces deux méthodes qui modifient le conteneur à partir d'un index, seront très pratiques pour programmer l'interface.

~\\ \question Programmez le reste des méthodes de l'interface \texttt{SortedSet<E>}. Pour vos tests, vous pouvez utiliser la classe de tests fournie avec le sujet du TP.

\begin{correction}
% Non vérifiée ;-)
{\color{red}
\begin{lstlisting}[language=Java]
    private int compare(E e1, E e2){
        if (comparator != null)
            return comparator.compare(e1, e2);
        return ((Comparable<E> e1).compareTo(e2);
    }
\end{lstlisting}
}

\end{correction}

\section{Interface \texttt{Map}}
Les implémentations de \texttt{Map} permettent de retrouver rapidement
une \texttt{Value} à partir de sa clef : l'association privilégie donc
un sens. Nous voudrions dans ce TP, avoir une association
bidirectionnelle. En utilisant toujours la notion de \texttt{Map} et
son implémentation \texttt{HashMap}, nous
allons implémenter l'interface \texttt{BidirectionnalMap} suivante:

\begin{verbatim}
/**
 * A BidirectionnalMap associates a primary with a secondary key  
 * but also the secondary with the primary key.
 * K1 and K2 are the types for primary and secondary key.
 */
public interface BidirectionnalMap<K1, K2> {
    public abstract K2 getFromPrimary(K1 k);
    public abstract K1 getFromSecondary(K2 k);

    public abstract void put(K1 k1, K2 k2);
    public abstract boolean isEmpty();
    public abstract void clear();
    public abstract void removeFromPrimary(K1 k1);
}
\end{verbatim}

Cette interface est disponible dans les ressources du TP sur moodle.

~\\ \question Quels sont les attributs nécessaires pour la
classe \texttt{Dictionnary} qui implémente l'interface
\texttt{BidirectionnalMap} ? Créez votre projet Eclipse, et la classe
\texttt{Dictionnary} avec ses attributs (Pensez à préciser que
\texttt{Dictionnary} implémente \texttt{BidirectionnalMap} dès la
création).

~\\ \question Programmez toutes les méthodes héritées de
\texttt{BidirectionnalMap} et redéfinissez la méthode
\texttt{toString}, de façon à ce qu'un appel à
\texttt{System.out.println(myDico)} affiche le contenu du dictionnaire
un couple par ligne.

~\\ \question Écrivez un petit programme qui permette de
traduire une phrase passée comme argument sur la ligne de commande. Vous aurez bien sûr à préalablement remplir votre dictionnaire avec les mots nécessaires.

\end{document}

