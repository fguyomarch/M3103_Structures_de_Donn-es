\documentclass[iutinfo, a4paper, nocorrections, 10pt]{ustl-tdtp}
%\usepackage[utf8]{inputenc}  
%\usepackage[T1]{fontenc}
\usepackage{fontspec}
\usepackage{listings}  
\usepackage{version}
\usepackage{tikz}


%\usepackage[a4paper]{geometry}


\etablissement{\ustl}
\formation{DUT info 2ème année}
\matiere{Structures de données}
\titre{TD : Révisions}
\date{\annee{2018}--\annee{2019}}
%\enseignant{}

%\includeversion{solution}
%\excludeversion{solution}

\lstset{
    breaklines=true
}

\newcommand{\case}{}%\rule[-.40cm]{0pt}{1cm}}
\newcommand{\ident}[3]{(\texttt{#1},~\texttt{#2},~\texttt{#3})}
\newcommand{\rien}{}

\parindent 0cm
\begin{document}
\maketitle
\thispagestyle{empty}



\subsection*{Exercice $\infty$ :Expressions arithmétiques}


\question Écrivez l'arbre binaire qui correspond à l'expression arithmétique infixe suivante : \\
\texttt{((7 + 3) - 1) + (2 * (4 - 5))}


\begin{solution}
{\color{red}

\begin{tikzpicture}
	\node (is-root) {+}
		[sibling distance=3cm]
		child {
			node {-}
				[sibling distance=2cm]
				child { 
					node {+} 
				[sibling distance=2cm]
				child{ node {7} }	
				child { node {3} }		
				}	
				child { node {1} }
				}	
		child {
			node {*}
				[sibling distance=2cm]
				child { 
					node {2} }	
				child { node {-} 
				  [sibling distance=2cm]
				  child{ node {4} }	
				  child { node {5} }
				}				
		};
		
\end{tikzpicture}
}
\end{solution}

~\\ \question Écrivez les versions préfixe et postfixe de l'expression.

\begin{solution}
{\color{red}
préfixe : + - + 7 3 1 * 2 - 4 5  \\
postfixe : 7 3 + 1 - 2 4 5 - * +

}
\end{solution}

~\\ \question Nous avons vu en cours comment évaluer une expression postfixe à partir d'une pile. En décrivant les états successifs de la pile, évaluez l'expression postfixe exprimée plus haut.

\begin{solution}
{\color{red}
entrée : \textbf{7} 3 + 1 - 2 4 5 - * +  \\
pile : 7 
\\
entrée : 7 \textbf{3} + 1 - 2 4 5 - * +  \\
pile : 7 3 
\\
entrée : 7 3 \textbf{+} 1 - 2 4 5 - * +  \\
pile : 10
\\
entrée : 7 3 + \textbf{1} - 2 4 5 - * +  \\
pile : 10 1
\\
entrée : 7 3 + 1 \textbf{-} 2 4 5 - * +  \\
pile : 9
\\
entrée : 7 3 + 1 - \textbf{2} 4 5 - * +  \\
pile : 9 2 
\\
entrée : 7 3 + 1 - 2 \textbf{4} 5 - * +  \\
pile : 9 2 4
\\
entrée : 7 3 + 1 - 2 4 \textbf{5} - * +  \\
pile : 9 2 4 5
\\
entrée : 7 3 + 1 - 2 4 5 \textbf{-} * +  \\
pile : 9 2 -1
\\
entrée : 7 3 + 1 - 2 4 5 - \textbf{*} +  \\
pile : 9 -2
\\
entrée : 7 3 + 1 - 2 4 5 - * \textbf{+}  \\
pile : 7
\\
}
\end{solution}

~\\ \question Écrivez la méthode \texttt{int evaluate(String [] input)} qui retourne le résultat de l'évaluation de l'expression postfixe contenu dans \texttt{input}(avec un élément de l'expression par case) à partir d'une pile. Vous considérerez qu'il n'y a que des opérandes entières ou des opérateurs binaires dans l'entrée et que l'expression est correcte. Vous disposez des méthodes suivantes:
\begin{itemize}
\item \texttt{int compute(int op1, int op2, String operator)} qui retourne le calcul de \texttt{op1 operator op2} 
\item \texttt{boolean isNumber(String s)} qui indique vrai si la chaîne de caractères \texttt{s} correspond à un nombre et faux sinon.
\item \texttt{Integer parseInt(String s)} qui convertit une chaîne de caractères \texttt{s} en entier.
\item \texttt{String valueOf(int i)} qui convertit un entier \texttt{i} en chaîne de caractères.
\end{itemize}.

\begin{solution}
{\color{red}
\begin{lstlisting}[language=Java]
public int evaluate(String [] input){
 Stack<String> stack = new Stack<String>();
  for(int i = 0; i < input.length; i++){
   String s = input[i];
    if(isNumber(s)){
     stack.push(s);
    }else{
      int op2 = Integer.parseInt(stack.pop());
      int op1 = Integer.parseInt(stack.pop());
      int res = compute(op1,op2,s);
      stack.push(String.valueOf(res));
    }
  }
  return Integer.parseInt(stack.pop());
}
\end{lstlisting}
}
\end{solution}

\question Écrivez une version récursive (sans pile) terminale de la méthode \texttt{evaluate}.

\begin{solution}
{\color{red}
\begin{lstlisting}[language=Java]
public int evaluate(String [] input){
  return evaluate(input,0);
}

private int evaluate(String [] input, int cursor){
 if(input.length==1) return Integer.parseInt(input[0]);
 else if(isNumber(input[cursor]))
  return evaluate(input,++cursor);
  else{ 
   int op1 = Integer.parseInt(input[cursor-2]);
   int op2 = Integer.parseInt(input[cursor-1]);
   int res = compute(op1,op2,input[cursor]);
   String [] s = new String[input.length-2];
   for(int i = 0, j = 0; i < input.length; i++){
     if(i<cursor-2||i>cursor){
	  s[j++] = input[i];
     }
     if(i==cursor){
      s[j++] = String.valueOf(res);
     }
   }
   return evaluate(s,0);
   }
}
\end{lstlisting}
}
\end{solution}




\end{document}

